%background, significance, statement of the problem
\chapter{Background}\label{chap:background}

China is famously known as a nation of tea drinkers. Coffee, the plant, however,
is no stranger to Chinese soil. Its earliest instance of cultivation was during
the 19th century in Yunnan province and large scale commercial plantation started
around the 1950s \autocite{zhang_coffee_2014}.

\section{Starting on Tea}

% they have provinces devoted to tea
% china has a tea drinking culture
% historically they have been engaging in tea trade
% there are ceremonies dedicated to tea
% how many types of tea are there in china? good proxy tell

To illustrate tea's importance to China, There are three aspects to consider:
\begin{enumerate*}
	\item China has been historically involved with tea,
	\item there are several provinces whose major industries involve tea, and
	\item tea has its own ceremony
\end{enumerate*}.

Tea is no stranger to legend. One story attributes its discovery to a legendary
emperor named Shen Nong. Legend has it that tea leaves had blown into his cup of
hot water and started from there \autocite[2]{fredholm_methylxanthines_2011}.
Another, known around Buddhist circles is the story of Bodhidharma, an
Indian monk who went to China and is known as the founder of Zen\footnote{Also
known as 禅 Chan} Buddhism. Upon his arrival, he immediately medidtated for nine
years, and at one point, fell asleep for a few moments. Determined to stay
awake, he cut off his eyelids in an effort not to fall asleep again. Upon seeing
this, Guanyin, the goddess of mercy, caused tea plants to grow on where
Bodhidharma's eyelids fell \autocite[2-3]{fredholm_methylxanthines_2011}.

During the Tang dynasty, tea was undergoing intensive cultivation. In the
eighth century, a scholar named \emph{Lu Yu} wrote the \emph{Cha jing},
translated as The Classic of Tea. detailing the history of tea. It also
describes the utensils needed to make tea and the process to make a good brew.
Through this document, tea was transformed from an ordinary drink into an art
form. Several industries flourished as a side effect of the massive demand for
tea, bringing in revenue in rural cash poor areas. A transport industry
developed, and wooden chests for containing tea grew. The porcelain industry was
spurred . Tea was then introduced to Japan and Korea after. Further cultivation
was done during the Song dynasty and a drinking habit was established and spread
throughout the country. Silk production grew in the north by the thirteenth
century to be used as barter for tea \autocite[2-10]{chow_all_1990}. By 1607 CE,
more than eight hundred years later, Europe experienced its first taste of tea
in Holland, Denmark by a ship that came in from Macau \autocite{chen_tea_2012}.

Tea was then introduced to Japan and Korea after. Further cultivation was done
during the Song dynasty and a drinking habit was established and spread
throughout the country. By 1607 CE, more than eight hundred years later, Europe
experienced its first taste of tea in Holland, Denmark by a ship that came in
from Macau \autocite{chen_tea_2012}.

\emph{Terroir} is a concept where ``the location is given special connotation,
where certain products are endowed with a unique identity that will influence
production and impart different impacts upon its final characteristics''
\autocite{silva_characterization_2014}. Depending on where tea is grown, its
quality, taste, and other characteristics is affected on where it is planted,
the type of soil the plant grows on, and the various climates it experiences
across locations. This is the concept that helps us differentiate Yunnan tea
from Fujian tea, barring specialized methods of preparing tea for consumption.

Yunnan, Fujian, Jiangsu, and Anhui are some of the many provinces in China that
cultivate tea. In fact, there are twenty provinces that produce tea, comprising
one-fourth of China's landmass \autocite{chen_tea_2012} and each known for a
speific specialty variant.

\section{The Market}

In 1999, the first Starbucks in the Mainland was established at the Beijing
World Trade Center When the idea of establishing Starbucks in Mainland China was
first brought up, there was skepticism on how could could a coffee shop chain
possibly compete against a country whose beverage market is dominated by tea and
whose drink prices are US\$3 per cup \autocite{adamy_eyeing_2006}?

% china has a tea culture
%substantiate

TODO: tea ceremony

\section{Open Door}

% then open door policy
% to economic boom

We move on to the modern era, during the time of Deng Xiaoping and the Open Door
Policy and known for the quote ``It doesn't matter if the cat is black or white, so
as long as it catches mice''. Economic reform was the plan and his leadership
was generally hands-off by nature, leaving decisions to officials he has
appointed, experts in their field, with Deng providing general guidance and
direction. For the rare times he has personally intervened for policymaking, his
decisions have proven cruicial and effective \autocite{naughton_deng_1993}.

For the initial steps of economic reform, the government loosened its grip on
existing enterprises. This left individual citizens now free to initiate their own
small to medium business ventures and letting these enterprising individuals dictate
their own prices and at quantities they wish to produce. This is in contrast to
Mao's plan of the state having control of all means of production throughout the
territory and controlled how much and how many products of a certain kind are
churned out. After which these products are sold to the government at basement
prices which are then resold to the general populace.

As an effect, private ownership became possible. Competition among firms heated,
stimulating development and creativity. This in turn incentivized workers and
increased productivity. The economy flourished.

% influx of FDI and SEZ
% with SEZ came in western brands
% then sbux

Then came the Special Economic Zones, or SEZs. A \emph{Special Economic Zone} is a
specially demarcated area, with a special administration handling that area.
Incentives like tax cuts are offered to businesses and enterprises when
established within the zone. On the state's part, procedures are streamlined to
make it easier and faster for these prospective businesses to get started.
An important provision of the SEZs is that foreign companies can establish
themselves there, with full ownership of their assets. Further, they are
exempted from import duties for selected items
\autocites{zeng_how_2011}{jaggi_chinas_1996}

When the program was first initiated in 1980, there initially four participating
cities across the two provinces of Guangdong and Fujian. These four cities were
Shenzen, Zhuhai, Shantou, and Xiamen. Hainan, Pudong, and other cities followed
after \autocite{jaggi_chinas_1996}.

% then who is starbucks
% starbucks is an important brand
%	from entry into chinese market, starbucks has been growing in china
%	ever since

\section{Where Starbucks fits in}

Then came Starbucks. While not the first, it is one of the many North American
food service chains to set foot early in China. Their initial foray into the Asian
market was in August 1996, whose first branch outside North America was
established in Tokyo. The plan was to form partnerships with local operators,
initially with a joint venture with Sazabu \autocite{harrison_exporting_2005}.
This was in the similar case of McDonald's in Japan in 1971, where the chain was
franchised by a then University of Tokyo student, Den Fujita.
\autocite[21, 113]{watson_golden_2006}

At China World Trade Building in Beijing, Starbucks made its first presence in
the Mainland in 1999. From one branch in 1999, they have expanded to 209 stores
in 2005, 470 stores at the turn of the decade in 2010. In 2011, Starbucks
planned to further increase their presence by opening 1000 more stores, with a
target of 1500 branches by 2015 \autocite{_starbucks_2011}. True to their word,
they have reached 1811 stores near the end of 2015
\autocite{statista_starbucks_2015}. CEO Howard Schulz is confident of the
company's progress within China so much that the corporation plans to open 500
more branches within the Mainland for the next five years
\autocite{burkitt_starbucks_2016}.


% add statista data at appendix

% significance
%	what does this mean
%	this might mean a possible and ongoing shift to chinese culture
%	from collective to individualitic and consumerist tendencies?
%	there is a change in culture
%	westernization of china?
%	western style food service is a sign of progress?
%	are they shedding their old beliefs in favor of the West or are they
%	appropriating it to their own


\section{Statement of the Problem}

Since Starbucks' entry into the cafe market in China in 1999, cafes have been a
hit with the population. This research desires to seek out the factors that
enabled Western-style cafes to be popular, with Starbucks as a specific example.
While being popular with the population, who are its patrons and what are their
motivations to frequent or even to just try out these cafes?

% where is it shifting from
% baseline habit?
% conservative old people to a new hedonistic lifestyle being headed by younger
% go list changes in habit
% generation
%	yes there is data under that age group

% there is a generational gap

% is there change?
% yes
% from where to where?
% what happens in between?
% will this continue?

% what are the factors that drove this change? is starbucks contributing to this
% new lifestyle?
% what factors increased that sense of self? starbucks being seen as a status
% symbol

% starbucks is a model to understand the shift
% the importance is a shift from tea to coffee. does coffee represent an
% individualist view? or maybe coffee as a western idea associated with
% individuaist tenencies

% why the habit is being formed, maybe there is something else aside from coffee
% to cafe, may be exposure to new media
% question may be as simple as china's shifting culture
% relate to the chinese dream

\begin{comment}
\begin{itemize}
	\item Habitus as a framework
	\item Drinker demographics
		% why do they drink coffee? is it really coffee they're after or
		% something else? looks like they're after something else
	\item How did international franchises adapt
	\item How did the coffee culture start
\end{itemize}
\end{comment}

% why starbucks? cafe culture is a new thing and how did this manage to get to
% this point?

% vim: textwidth=80 smartindent breakindent
