%background, significance, statement of the problem
\section{Background}\label{sec:background}

China is famously known as a nation of tea drinkers. Coffee, the plant, however,
is no stranger to Chinese soil. Its earliest instance of cultivation was during
the 19th century in Yunnan province. Large scale commercial plantation
started around the 1950s \autocite{zhang_coffee_2014}.
% find source on who brought it in

Coffee makes its appearance in Traditional Chinese Medicine as a medicinal herb
tasked in stimulating liver qi, purging the gallbladder, and regulating the
menstrual cycle \autocite{dharmananda_coffee_2003}.
\textcite{namba_historical_2001} recounts how coffee first arrived in Japan
through translations of Dutch textbooks. They further study how the framework of
how Chinese and Arabic traditional medicine dealt with and described the effects
of coffee, mostly through caffeine and its effects on aging, infectious
diseases, and heart processes.

Starbucks' initial foray into the Asian market was in August 1996, whose first
branch outside North America was in Tokyo. The plan was to form partnerships
with local operators, initially with a joint venture with Sazabu. Branches
proliferated with success since then, exceeding Starbucks' own predictions
\autocite{harrison_exporting_2005}. In 1999, Starbucks opened its first store in
Mainland China, at China World Trade Building, Beijing
\autocite{starbucks_corporation_history_????}.

\subsection{Statement of the Problem}

Given these items, Would Bourdieu's \emph{Habitus} be a suitable framework for
explaining the growing coffee culture in the country? Who are its primary
clientele? For what reasons do they drink coffee? Would they prefer to drink at
home or in a cafe?

\begin{itemize}
	\item Habitus as a framework
	\item Drinker demographics
	\item How did international franchises adapt
	\item How did the coffee culture start
\end{itemize}
