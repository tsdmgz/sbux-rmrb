\section{Review of Related Literature}

In analyzing coffee, \autocite{silva_characterization_2014} puts forward
\emph{terroir}, a concept used in winemaking to analyze its quality and taste.
Location, altitude, and soil type are variables in crop production and influence
taste profile of a specific type of coffee bean. The study managed to
efficiently determine differences with harvests among the locations sampled.

\autocite{su_impact_2006} studies the impact of western culture in Taiwan. Three
studies were performed, correlating consumption patterns with the desirability
of foreign culture and the ``adoring of foreign value in coffee consumption''.
It was found that foreign brands were being perferred over local brands for tea
and coffee. This lead to the conclusion that there was an adoration of western
culture and influenced consumers' purchase decisions and drink preferences.
\autocite{euromonitor_international_coffee_2015} illustrates the growing coffee
trend in Taiwan. Sales of instant coffee, Nestlé to be specific, dominate the
market. In addition, a new format of coffee preparation was introduced to the
market in the form of ``coffee pods'' through the brand ``Nespresso'', a
division of Nestlé. However, there is a growing demand for fresh coffee beans
driven by an increasing number of cafés opening up in Taiwan. Consumers have
started to gain access to coffee making equipment, enabling them to prepare
their own serving at home. Because of an increased demand of for fresh coffee,
prices have risen due to consumers seeking better quality coffee. As a net
effect, the rise of coffee has cannibalized tea sales and consumption, with
consumers preferring ready-to-drink tea mixes while on the go.

In China, \autocite{harrison_exporting_2005} describes Starbucks' initial foray
into the international market, along with descriptions of its business model.
For example, in an effort to stimulate consumption, Starbucks places several
stores nearby other branches. \autocite{yang_consumer_2012} measured how much of
a premium, consumers in Wuhan are willing to pay for coffee that is labelled as
``Fair Trade''. It was found that consumers were willing to pay as much as 22\%
more on average than traditional coffee. While not mentioned in the study,
Starbucks promotes its products as fair trade goods.
