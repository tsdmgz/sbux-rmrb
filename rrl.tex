\section{Review of Related Literature}

\autocite{su_impact_2006} studies the impact of western culture in Taiwan. Three
studies were performed, correlating consumption patterns with the desirability
of foreign culture and the ``adoring of foreign value in coffee consumption''.
It was found that foreign brands were being perferred over local brands for tea
and coffee. This lead to the conclusion that there was an adoration of western
culture and influenced consumers' purchase decisions and drink preferences.
\autocite{euromonitor_international_coffee_2015} illustrates the growing coffee
trend in Taiwan. Sales of instant coffee, Nestlé to be specific, dominate the
market. In addition, a new format of coffee preparation was introduced to the
market in the form of ``coffee pods'' through the brand ``Nespresso''. However,
there is a growing demand for fresh coffee beans driven by an increasing number
of cafés opening up in Taiwan. Consumers have started to gain access to coffee
making equipment, enabling them to prepare their own serving at home. Because of
an increased demand of for fresh coffee, prices have risen due to consumers
seeking better quality coffee. As a net effect, the rise of coffee has
cannibalized tea sales and consumption, with consumers preferring ready-to-drink
tea mixes while on the go.
