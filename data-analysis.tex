\chapter{Data and Analysis}

In the article ``Chinese prefer foreign brands'', a market survey of the top
1000 brands by Campaign
Asia-Pacific found that most recognized brands in China are from other
countries. The top ten brands were Apple Inc., Nestlé, Chanel, Sony, Samsung.
Uni-President Enterprises Corporation\footnote{Company is Taiwan based},
Panasonic, Nike, Canon, and Starbucks. It is interesting to note that there is
none of these recognized brands are based in the mainland. The first mainland
company in the list would be Beijing Tong Ren Tang, a pharmacy chain at the
eleventh spot in the list.

Comparatively, not much has changed in four years. In a 2016 report of the
same survey, the top ten brands were Samsung, Nestlé, Chanel, Apple, Sony, Nike,
Beijing Tong Ren Tang, Starbucks, Adidas, and Panasonic. While other brands
merely shuffled positions, there were only two new entrants to the top ten,
namely Beijing Tong Ren Tang and Adidas. Only one of which is a local brand,
Beijing Tong Ren Tang.

% write on western culture adoration? west as a model of success or something

Within the dataset, there were instances where a chain was criticized for
opening in a specific location and left alone in another. One of these
controversial locations was at Lingyin temple\footnote{``Buddhist Starbucks
stirs Controversy''}, a Buddhist temple first established during the fourth
century. As reported in the article, a Starbucks branch had opened up nearby and
had gathered controversy from social media users over at Weibo and news
commentators alike, even resulting in a few editorial
articles\footnote{``Controversy progresses on Starbucks' tying knot with Lingyin
temple'' and ``Keeping historical sites away from mercenary stinks''.}. Another
instance where a fast food establishment generated controversy was that of a
McDonald's branch that was set up in the home of a historical Kuomintang
official. Again, comments from social media ranged from annoyance to anger.

While it is understandable that when a historical site has been modified, it
will generate some form of controversy. What is notable here are the arguments
used. A common theme among these was of ``cultural invasion'' and ``purity''. It
is argued that such chains such as McDonald's and Starbucks was slowly
eroding the culture of China.

On one article, ``Smoother, faster ride home for Spring Festival'', trains and
infrastructure were being modernized. New amenities like onboard WiFi and
internet-based ticket sales systems were installed. Standing out was there was a
specific mention of Starbucks coffee being served on board.  Another article
stating that a new mall was being constructed in Urumqi will have the latest
theatrical setup, with international botiques and chains such as Uniqlo,
Watson's, McDonald's, and Starbucks branches initially seeding its stores. A
2012 article reported that Nanjing had a hospital constructed with ``five-star
facilities'', with a RMB 7 million piano and a Starbucks branch. All these
without more than a paragraph or two in the article.

Basing from these reactions, limits are being delineated. It seems that
Western style establishments cannot be near of be established at a ``cultural''
site. In contrast, these establishments are are loved and delighted upon when
established within an urbanized area.

Further, a characteristic pattern of writing stood out. ``Western'' is
frequently written with a capital ``W''. While the difference between
capitalization seems innocuous, there are connotations at play with ``Western''
and its capital W. Ordinarily, this describes a direction, however here is
frequently used to describe something modern, advanced, and desirable.

% there is a conflicting attitude with these establishments: these are wildly
% raved over and yet it is also a cultural invasion?
% is there tension?
% contrast this with say, vigan and heritage walk? what if the chains were
% locally owned?
% despite pushback on certain areas, development continues
% "Western" really has a strong connotation

% this isn't a case of colonial mentality. China wasn't colonized totally.

% big name brands, and luxury at that

% It is said that imitation is the highest form of flattery. In the case of Apple,
% the imitation they have experienced is beyond flattery. Their flagship device,
% the iPhone, is widely counterfeited with varying degrees of quality hoping to
% cash in to unsuspecting and impatient buyers wanting to get their hands on the
% latest device earlier than the rest \autocite{justice_iphone_2015}. The imitation
% does not end at devices. Even the Apple Store itself is copied by enterprising
% entrepreneurs also hoping to cash in on the Apple craze. These lookalike stores
% have managed to imitate the look, form, and feel of an original Apple Store down
% to uniform and store layout \autocite{lee_chinas_2015}.

\section{Conclusion}

Prior literature stated that the main success of Starbucks within China was the
allure of its brand. It is associated as a status symbol of individualism and of
being a modern person. Upon analysis, patterns emerged from these articles.
There are strong connotations of an ``us and them'', where the Western is them,
and this is us, which is being looked upon as a model of development. However,
as much as these establishments are hyped, clear limits are made on where they
can and cannot establish. Cities, towns, and generally urbanized areas are not
seen as a problem and historical, or ``cultural'' areas are off-limits.

% looking to the west, specifically, to the united states as a form of
% development. these stores wouldn't have come here if they don't see potential
% sales anyway so that means there is approval by the chinese public

% A recurring theme among these studies\footnote{TODO: cite these things} is the
% concept of modernity and progress being synonymous with The West, with the
% United States as the model. Despite the ongoing rhetoric between the two
% governments\footnote{See \textcite{dizon_south_2016} for an example}, there is
% significant consumption of cultural and technological products exported by the
% United States as seen with the extreme popularity of Apple products and strong
% presence of other fast food chains like McDonald's and KFC.
%
% Starbucks, in similar fashion with other brands like Apple, enjoy a cult
% following among its fans.

% they are trying to integrate themselves further into the chinese lifestyle

% vim: textwidth=80 smartindent breakindent syntax=tex
