\chapter{Data and Analysis}

Having established that tea has strong cultural roots with the Chinese, let us
take a look at how Chinese see coffee.

It is said that imitation is the highest form of flattery. In the case of Apple,
the imitation they have experienced is beyond flattery. Their flagship device,
the iPhone, is widely counterfeited with varying degrees of quality hoping to
cash in to unsuspecting and impatient buyers wanting to get their hands on the
latest device earlier than the rest \autocite{justice_iphone_2015}. The imitation
does not end at devices. Even the Apple Store itself is copied by enterprising
entrepreneurs also hoping to cash in on the Apple craze. These lookalike stores
have managed to imitate the look, form, and feel of an original Apple Store down
to uniform and store layout \autocite{lee_chinas_2015}.

% looking to the west, specifically, to the united states as a form of
% development. these stores wouldn't have come here if they don't see potential
% sales anyway so that means there is approval by the chinese public

A recurring theme among these studies\footnote{TODO: cite these things} is the
concept of modernity and progress being synonymous with The West, with the
United States as the model. Despite the ongoing rhetoric between the two
governments\footnote{See \textcite{dizon_south_2016} for an example}, there is
significant consumption of cultural and technological products exported by the
United States as seen with the extreme popularity of Apple products and strong
presence of other fast food chains like McDonald's and KFC.

% Starbucks, in similar fashion with other brands like Apple, enjoy a cult
% following among its fans.

% vim: textwidth=80 smartindent breakindent
