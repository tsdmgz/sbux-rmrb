\section{Data and Analysis}

Having established that tea has strong cultural roots with the Chinese, let us
take a look at how Chinese see coffee.

In places where coffee is an established drink, its consumption is recognized as
a drink to wake someone up or to keep someone awake when sleepy. For the
Chinese, tea fulfills that role, as used by Buddhist monks to keep themselves
awake during long meditation sessions \autocite[268]{kieschnick_impact_2003}.
Another source of tea's solid foundation in China is Lu Yu's classical work
called Chajing (茶经), translated as The Classic of Tea
\autocite{lu_classic_1974}.  Described also is how tea and tea drinking spread
throughout China \autocite[266-267]{kieschnick_impact_2003}. The work describes
how tea should ideally be prepared, boiled, and served.

As for today, tea is still the dominant beverage in China, and in a large part
of the world \autocite{that_euromon_stat_on_tea}. From specialty variants like
Tieguanyin of Fujian and Taiwan to Pu'er of Yunnan, 
% ok so we're on our way to establish the habit of drinking tea

% tea takes its place, as seen in buddhist stories where monks drank tea to keep
% them up during long meditation sessions

% but coffee is new here. it is a ``western'' drink. it is shiny and new.
% coffee is seen as the west, progress, and all

It is said that imitation is the highest form of flattery. In the case of Apple,
the imitation they have experienced is beyond flattery. Their flagship device,
the iPhone, is widely counterfeited with varying degrees of quality hoping to
cash in to unsuspecting and impatient buyers wanting to get their hands on the
latest device earlier than the rest \autocite{justice_iphone_2015}. The imitation
does not end at devices. Even the Apple Store itself is copied by enterprising
entrepreneurs also hoping to cash in on the Apple craze. These lookalike stores
have managed to imitate the look, form, and feel of an original Apple Store down
to uniform and store layout \autocite{lee_chinas_2015}.

% looking to the west, specifically, to the united states as a form of
% development. these stores wouldn't have come here if they don't see potential
% sales anyway so that means there is approval by the chinese public

A recurring theme among these studies\footnote{TODO: cite these things} is the
concept of modernity and progress being synonymous with The West, with the
United States as the model. Despite the ongoing rhetoric between the two
governments\footnote{See \textcite{dizon_south_2016} for an example}, there is
significant consumption of cultural and technological products exported by the
United States as seen with the extreme popularity of Apple products and strong
presence of other fast food chains like McDonald's and KFC.

% vim: textwidth=80 smartindent breakindent
