\chapter{Review of Related Literature}\label{chap:rrl}

While other countries have coffee as their perk-me-up drink of choice, China has
tea for their case. This is evident with ready-to-drink (RTD) tea sales
specially in the retail sector
\autocite{international_coffee_council_coffee_2015}. The same holds true in a
large part of the world, and will be so for a good period of time
\autocite{CITEME}. In places where coffee holds dominance, it serves the same
purpose as tea would: as a recreational drink and as something to keep one
awake.

% For the Chinese, tea fulfills that role, as used by Buddhist monks to keep
% themselves awake during long meditation sessions
% \autocite[268]{kieschnick_impact_2003}.  Another source of tea's solid
% foundation in China is Lu Yu's classical work called Chajing (茶经),
% translated as The Classic of Tea \autocite{lu_classic_1974}, describing tea's
% history through the time of Lu Yu. Described also
% is how tea and tea drinking spread throughout China
% \autocite[266-267]{kieschnick_impact_2003}. The work describes how tea should
% ideally be prepared, boiled, and served.

Tea also appears in other texts, through its close connections with Buddhism.
\textcite{kieschnick_impact_2003} details how tea came into regular use by monks
as a drink to keep their concentration going during long meditation sessions. He
also describes here tea's various appearances in legends and ancient writing
such as \textcite{lu_classic_1974} on The Classic of Tea.

In analyzing coffee, \textcite{silva_characterization_2014} puts forward
\emph{terroir}, a concept used in winemaking to analyze its quality and taste.
Location, altitude, and soil type are variables in crop production and influence
taste profile of a specific type of coffee bean. The study managed to
efficiently determine differences with harvests among the locations sampled.



For health, \textcite{dharmananda_coffee_2003} analyzes coffee in the context of
Traditional Chinese Medicine. The coffee plant and its relatives have
historically been a source of medicinal herbs. Coffee in particular has an
effect of regulating liver qi, giving the experience of ``a strong sense of
mental and physical vitality''. \textcite{namba_historical_2001} recounts how
coffee first arrived in Japan through translations of Dutch textbooks. They
further study how the framework of how Chinese and Arabic traditional medicine
dealt with and described the effects of coffee, mostly through caffeine and its
effects on aging, infectious diseases, and heart processes.

\textcite{su_impact_2006} studies the impact of western culture in Taiwan. Three
studies were performed, correlating consumption patterns with the desirability
of foreign culture and the ``adoring of foreign value in coffee consumption''.
It was found that foreign brands were being perferred over local brands for tea
and coffee. This lead to the conclusion that there was an adoration of western
culture and influenced consumers' purchase decisions and drink preferences. A
market research study by \textcite{euromonitor_international_coffee_2015}
illustrates the growing coffee trend in Taiwan. Sales of instant coffee, Nestlé
to be specific, dominate the market. In addition, a new format of coffee
preparation was introduced to the market in the form of ``coffee pods'' through
the brand ``Nespresso'', a division of Nestlé. However, there is a growing
demand for fresh coffee beans driven by an increasing number of cafés opening up
in Taiwan. Consumers have started to gain access to coffee making equipment,
enabling them to prepare their own serving at home. Because of an increased
demand of for fresh coffee, prices have risen due to consumers seeking better
quality coffee. As a net effect, the rise of coffee has cannibalized tea sales
and consumption, with consumers preferring ready-to-drink tea mixes while on the
go.

In China, \textcite{harrison_exporting_2005} describes Starbucks' initial foray
into the international market, along with descriptions of its business model.
For example, in an effort to stimulate consumption, Starbucks places several
stores nearby other branches. \textcite{yang_consumer_2012} measured how much of
a premium, consumers in Wuhan are willing to pay for coffee that is labelled as
``Fair Trade''. It was found that consumers were willing to pay as much as 22\%
more on average than traditional coffee. While not mentioned in the study,
Starbucks promotes its products as fair trade goods.

\textcite{watson_golden_2006} presents an ethnographic case study of the fast
food chain McDonald's across five cities: Beijing, Hong Kong, Seoul, Taipei, and
Osaka. The study describe locals' perception of McDonald's in each of these
cities. These perceptions range from being a window to the world outside of
China in the case of Beijing, to being a political act and a reflection of
attitude on the subject of reuinifcation in Taiwan, and to the eventual
integration of everyday culture in Japan.

\textcite{zhang_coffee_2014} argues that coffee drinking in China is more of a
trend than a habit. Most of its consumers are of a younger and more affluent
demographic. Drinkers are found to be concentrated among major cities like
Beijing, Shanghai, and Guangzhou.
\textcite{international_coffee_council_coffee_2015} notes that there are no
clear statistics on China's coffee drinkers, but consumption can be estimated by
import and export volumes. Coffee imports are noted to be mostly in unprocessed
and non-roasted form. While Yunnan(云南)is an area known for tea, coffee
production is rising.

% vim: textwidth=80 smartindent breakindent
