\chapter{Data and Analysis}

In the article ``Chinese prefer foreign brands'', a market survey of the top
1000 brands by Campaign
Asia-Pacific found that most recognized brands in China are from other
countries. The top ten brands were Apple Inc., Nestlé, Chanel, Sony, Samsung.
Uni-President Enterprises Corporation\footnote{Company is Taiwan based},
Panasonic, Nike, Canon, and Starbucks. It is interesting to note that there is
none of these recognized brands are based in the mainland. The first mainland
company in the list would be Beijing Tong Ren Tang, a pharmacy chain at the
eleventh spot in the list.

Comparatively, not much has changed in four years. In a 2016 report of the
same survey, the top ten brands were Samsung, Nestlé, Chanel, Apple, Sony, Nike,
Beijing Tong Ren Tang, Starbucks, Adidas, and Panasonic. While other brands
merely shuffled positions, there were only two new entrants to the top ten,
namely Beijing Tong Ren Tang and Adidas. Only one of which is a local brand,
Beijing Tong Ren Tang.

% write on western culture adoration? west as a model of success or something

%%%% pattern on historical sites

Within the dataset, there were (TODO:two|three) instances where a chain was
criticized for opening in a specific location and left alone in another. One of
these controversial locations was at Lingyin temple, a Buddhist temple first
established during the fourth century. As reported in the article, a Starbucks
branch had opened up nearby and had gathered controversy from social media users
over at Weibo and news commentators alike.

% there is a conflicting attitude with these establishments: these are wildly
% raved over and yet it is also a cultural invasion?
% is there tension?
% contrast this with say, vigan and heritage walk? what if the chains were
% locally owned?
% despite pushback on certain areas, development continues
% "Western" really has a strong connotation

% this isn't a case of colonial mentality. China wasn't colonized totally.

% big name brands, and luxury at that

It is said that imitation is the highest form of flattery. In the case of Apple,
the imitation they have experienced is beyond flattery. Their flagship device,
the iPhone, is widely counterfeited with varying degrees of quality hoping to
cash in to unsuspecting and impatient buyers wanting to get their hands on the
latest device earlier than the rest \autocite{justice_iphone_2015}. The imitation
does not end at devices. Even the Apple Store itself is copied by enterprising
entrepreneurs also hoping to cash in on the Apple craze. These lookalike stores
have managed to imitate the look, form, and feel of an original Apple Store down
to uniform and store layout \autocite{lee_chinas_2015}.

% looking to the west, specifically, to the united states as a form of
% development. these stores wouldn't have come here if they don't see potential
% sales anyway so that means there is approval by the chinese public

% A recurring theme among these studies\footnote{TODO: cite these things} is the
% concept of modernity and progress being synonymous with The West, with the
% United States as the model. Despite the ongoing rhetoric between the two
% governments\footnote{See \textcite{dizon_south_2016} for an example}, there is
% significant consumption of cultural and technological products exported by the
% United States as seen with the extreme popularity of Apple products and strong
% presence of other fast food chains like McDonald's and KFC.
%
% Starbucks, in similar fashion with other brands like Apple, enjoy a cult
% following among its fans.

% they are trying to integrate themselves further into the chinese lifestyle

% vim: textwidth=80 smartindent breakindent syntax=tex
