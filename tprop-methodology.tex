\chapter{Methodology}\label{chap:methodology}

I will attempt to answer these questions using Bourdieu's framework of
\emph{Habitus}. Articles from People's Daily (English) have been gathered for
this study. These articles will then be subject to content analysis, on the
lookout for recurrent themes.

Being a daily publication, newspapers provide a good form of coverage for small
incidents and important events. A newspaper provides the role of an archival
chronicle, keeping track of what was happening on a day to day basis
\autocite{martin_examining_1996}. People's Daily was selected because of its
wide circulation in China, enjoying significant readership and wide reach across
the population. Public officials and ministers have chosen to release official
statements through this outlet.

In gathering articles, a simple search with the term "Starbucks" within the
\href{http://en.people.cn}{People's Daily
website}\footnote{\url{http://en.people.cn}} was performed, returning 81
articles, the earliest of which date back to August 2007. A small Python script
(see appendix \autoref{appdx:grab-link}) was written to programatically download
all matching articles. While all news articles presented in the search results
dating back from 2007 were downloaded for the dataset, analysis was
intentionally limited to stories within January 2010 until May 2016. Refining
further, stories that did not happen within China's borders are not included in
the analysis. A complete archive of all downloaded articles and this paper
itself are available in a \href{https://github.com/tsdmgz/sbux-rmrb}{GitHub
repository}\footnote{\url{https://github.com/tsdmgz/sbux-rmrb}}. Sample articles
that were included and excluded can be found in appendices
\ref{appdx:news-articles-inc} and \ref{appdx:news-articles-ninc}.

\section{Analytical Framework}\label{sec:analyticf}

% Data sources include statistical studies of coffee drinkers in China,
% news articles from major outlets, and press releases from companies
% themselves. Thematic analysis was applied, looking for themes with regard to
% consumption patterns of patrons, location and placement of stores,

\section{On the case of Starbucks}\label{sec:case-of-sbux}

One may question, so why Starbucks? There are a number of competing coffee
chains in China knowing that Starbucks was not the first of its kind in the
Mainland.

There is the popularity aspect. Yes, there are other coffee chains like
Taiwanese-based UBC
Coffee \autocite{CITEME}, and Costa Coffee from Hong Kong \autocite{CITEME}, but
Starbucks is not known as yet another coffee chain, but it is rather known as a
famous coffee chain. They are a place to be that sells coffee, rather than
something that sells coffee that offers itself as a place to be.

% how commercial coffee first got to china?
% the cafe culture is a new idea to them (how do you say that?)
% why newspaper? because it is one of the record of events (possibly canonical)
% of a society

% vim: textwidth=80 smartindent breakindent syntax=tex
