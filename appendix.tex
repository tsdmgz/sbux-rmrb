\appendix
\chapter{article-grab-link.py}\label{appdx:grab-link}

\inputminted[linenos]{python}{data/script/article-grab-link.py}

\section{Usage}
% Thank you POSIX manpages for content inspiration

\subsection{Synopsis}

article-grab-link.py -t PUBLICATION -f FILE

\subsection{Description}

Parses a downloaded copy of a search results page of a given FILE from
PUBLICATION and returns links to articles in the given FILE.

\subsection{Options}

\begin{description}
	\item [-t PUBLICATION] Use PUBLICATION in parsing a saved results page.
	\item [-f FILE] Obtain links from FILE. Multiple input files are not
	supported.
\end{description}

\subsection{Use case}

In the process of grabbing the articles, a search is performed in People's Daily
website which then returns a results page with the URL
\begin{minted}[breaklines=true]{text}
http://search.people.com.cn/language/english/getResult.jsp
\end{minted}
However, this is not the page we are looking for. The actual URL for the results
page is at the bottom part, at the pagination links and looks like
\begin{minted}[breaklines=true]{text}
http://search.people.com.cn/language/search.do?pageNum=3&keyword=starbucks
&siteName=english&dateFlag=true&a=&b=&c=&d=&e=&f=
\end{minted}

The URL for the results page is now known. As of writing, there are 9 pages in
the results page. A command to download all the pages in the search results can
now be constructed.
% TODO: figure out how to unbreak this somehow
\begin{minted}[breaklines=true]{bash}
cd data/articles/peoples-daily/searchresults
echo "http://search.people.com.cn/language/search.do?pageNum="{1..9}
"&keyword=starbucks&siteName=english&dateFlag=true&a=&b=&c=&d=&e=&f="
 | parallel -n1 -j9 wget -v
\end{minted}
With all pages from the search results downloaded, links to resulting articles
can now be parsed.
\begin{minted}[breaklines=true]{bash}
cd articles
find ../ -maxdepth 1 -type f|parallel ../../../script/article-grab-link.py -t rmrb -f {}|parallel -j30 wget -v {}
\end{minted}
All downloaded articles are downloaded in the current working directory. If the
commands were followed, look in
\mintinline{bash}{data/searchresults/peoples-daily/articles}.

% vim: textwidth=80 smartindent breakindent
