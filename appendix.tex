\appendix
\chapter{article-grab-link.py}\label{appdx:grab-link}

\inputminted[linenos]{python}{data/script/article-grab-link.py}

\section{Usage}
% Thank you POSIX manpages for content inspiration

\subsection{Synopsis}

article-grab-link.py -t PUBLICATION -f FILE

\subsection{Description}

Parses a downloaded copy of a search results page of a given FILE from
PUBLICATION and returns links to articles in the given FILE.

\subsection{Options}

\begin{description}
	\item [-t PUBLICATION] Use PUBLICATION in parsing a saved results page.
	\item [-f FILE] Obtain links from FILE. Multiple input files are not
	supported.
\end{description}

\subsection{Use case}

In the process of grabbing the articles, a search is performed in People's Daily
website which then returns a results page with the URL
\begin{minted}[breaklines=true]{text}
http://search.people.com.cn/language/english/getResult.jsp
\end{minted}
However, this is not the page we are looking for. The actual URL for the results
page is at the bottom part, at the pagination links and looks like
\begin{minted}[breaklines=true]{text}
http://search.people.com.cn/language/search.do?pageNum=3&keyword=starbucks
&siteName=english&dateFlag=true&a=&b=&c=&d=&e=&f=
\end{minted}

The URL for the results page is now known. As of writing, there are 9 pages in
the results page. A command to download all the pages in the search results can
now be constructed.
% TODO: figure out how to unbreak this somehow
\begin{minted}[breaklines=true]{bash}
cd data/articles/peoples-daily/searchresults
echo "http://search.people.com.cn/language/search.do?pageNum="{1..9}
"&keyword=starbucks&siteName=english&dateFlag=true&a=&b=&c=&d=&e=&f="
 | parallel -n1 -j9 wget -v
\end{minted}
With all pages from the search results downloaded, links to resulting articles
can now be parsed.
\begin{minted}[breaklines=true]{bash}
cd articles
find ../ -maxdepth 1 -type f|parallel ../../../script/article-grab-link.py -t rmrb -f {}|parallel -j30 wget -v {}
\end{minted}
All downloaded articles are downloaded in the current working directory. If the
commands were followed, look in
\mintinline{bash}{data/searchresults/peoples-daily/articles}.

\chapter{Sample included articles}\label{appdx:news-articles-inc}

\section{Former Kuomintang Leader's Old Residence Turned into
McDonald's}

\begin{itemize}
	\item Author: not listed
	\item Editor: Liang Jun, Bianji
	\item Syndicated: Yes, CRI Online
\end{itemize}

\begin{displayquote}

	Former Kuomintang leader Chiang Ching-kuo's former residence in east
	China's Hangzhou city has been turned into a McDonald's fast food
	outlet.

	The McDonald's outlet has opened in the main building of the old
	residence, a western-style brick and wood villa built in the 1930s.

	A Starbucks cafe opened two months earlier this year, in a side wing of
	the same building, next to the McDonald's cafe.

	Chiang Ching-kuo was a Kuomintang politician and leader, and son of
	Chiang Kai-shek, who held numerous leadership positions in Taiwan.

	The use of Chiang Ching-kuo's old residence as a coffee shop and fast
	food outlet has stirred concerns about the protection of historical
	sites among netizens.

	An expert on historic preservation says developing historical sites for
	commercial purposes and conducting renovation should both need
	government approval.

	The expert added that those renovating historical sites are not allowed
	to change any interior room structure or exterior building walls.

\end{displayquote}

\subsection{Reason}

Another Starbucks branch has been opened in a historical/cultural site and has
generated controversy, again.

\section{Smoother, faster ride home for Spring Festival}

\begin{itemize}
	\item Author: not listed
	\item Editor: Ma Xiaochun, Bianji
	\item Syndicated: Yes, Xinhua
\end{itemize}

\begin{displayquote}

	BEIJING - High-speed trains with comfort, Starbucks coffee onboard, free
	WIFI in stations, and phone apps for ticket purchase. As the Spring
	Festival travel rush kicked off on Sunday, hundreds of millions of
	Chinese found that their journeys for holiday homecomings have become
	much smoother and faster.

	This year's Spring Festival travel rush reflected how China's economic
	boom, huge investment in infrastructure and fast growth of information
	technologies totally redefined the once gruelling experience of going
	home for the Chinese New Year, which falls on February 8 this year.

	MODERNIZED JOURNEY

	At Shanghai Railway Station, the ticket office is no longer crowded. In
	previous years' travel rush, the ticket office was crammed every night
	with tens of thousands of people who had to line up for the whole night
	to buy a ticket.

	But this year, about 83 percent of tickets were purchased online.

	China's railway service has been adapting to hi-tech trends by making
	itself accessible through websites and mobile phone apps, said Zhu
	Wenzhong, passenger traffic director of Shanghai Railway Bureau.
	Passengers now could order onboard meals on the phone app before
	boarding. Drinks made by Starbucks are available on certain trains.

	Across China, free WIFI is offered in some train stations and electronic
	ticketing machines were placed in bus stations. An online system that
	integrates bus operators in 13 provinces has been launched.

	The Ministry of Transport said this year it started to use big data to
	analyze the Spring Festival traffic.

	Chinese car-hailing app Didi rolled out a car-pooling service that can
	pair travelling needs across the country, making it possible for drivers
	to take on others when travelling home for the Chinese Lunar New Year.

	Train stations have also been modernized. In the city of Nanchang, a
	railway hub in east China, passengers used to wait outside Nanchang
	Railway Station as there was not enough room indoors during the Spring
	Festival travel rush. But this year, they can wait inside the station as
	a high-speed train station was just added to the city.

	FASTER RIDE

	This year, Gong Xinyi, a college student in Shanghai, traveled back to
	her hometown in Jiangxi Province with only one third of the time that
	she used to spend.

	A newly added high-speed route has linked Gong's hometown with Shanghai
	and shortened her journey to three hours. Last year, she had to take a
	7-hour-train ride and an additional 3-hour bus trip.

	Gong's faster Spring Festival journey is made possible as China has been
	investing heavily to expand its high-speed train network which is
	already the world's largest.

	Of all the trains serving in the Spring Festival travel rush this year,
	more than 60 percent are high-speed trains that can run up to 350
	kilometers per hour.

	Around 3,300 kilometers of new lines were added to the high-speed
	railway network last year, bringing the total length to 19,000
	kilometers, which make up 60 percent of the world's total.

	From 2011 to 2015, the period in which China's 12th Five-Year Plan was
	implemented, fixed-asset investment in railways amounted to 3.58
	trillion yuan (544 billion U.S. dollars), up 47.3 percent from the 11th
	five-year-plan period.

	Sheng Guangzu, general manager of the China Railway Corp., said China
	plans to invest 800 billion yuan in railways in 2016, especially in
	less-developed central and western regions.

	High-speed rail service continues to carry more weight in the Spring
	Festival travel rush because more Chinese now can afford to travel in
	style after the country's average disposable income surged by more than
	seven percent every year over the last decade.

	Chinese people's growing ability to afford a faster journey has also
	fueled an air travel boom.

	Chinese airlines are expected to carry 54.55 million passengers in the
	Spring Festival travel rush, up 11 percent from the last year.

	Air China said during the Spring Festival travel rush, it would add
	2,432 flights and operate an average of 1,160 flights daily.

	China Southern Airlines planned to add over 6,000 flights on 155
	international and domestic routes during the Spring Festival travel
	rush.

	Effort has also been made to ensure a faster trip back on the ground. As
	no toll way fee is charged nationwide during the seven-day Spring
	Festival holiday, traffic jams on highways had trapped thousands of cars
	for hours in previous years.

	Zheng Zongjie, an engineer at the road network center of the Ministry of
	Transport, said this year car drivers would no longer need to pick up
	the tickets at toll gates to make traffic smoother, as they did during
	previous free-hours.

	Urban transportation would also be improved with better arrangement of
	metro, bus and taxi in the Spring Festival travel rush, the authorities
	said.

\end{displayquote}

\subsection{Reason}

Aside from adopting new tech innovations, it is interesting to note that
Starbucks coffee has been specifically mentioned, when a coach may also have
other food and/or beverage products on board

\chapter{Sample excluded articles}\label{appdx:news-articles-ninc}

\section{Lack of ``excellent'' coffee blends: Consumer Reports}

\begin{itemize}
	\item Author: not listed
	\item Editor: not listed
	\item Syndicated: China Daily/Agencies
\end{itemize}

\begin{displayquote}

	After tasting 37 different blended coffees, Consumer Reports couldn't
	find one that measured up to its ``excellent'' or ``very good'' ratings,
	the publication said Tuesday.

	The less-than-glowing report follows a year that saw tight supplies of
	high-quality arabica coffee beans in Colombia, followed by steep
	premiums that caused some roasters to look for cheaper and more
	available options for their blends.

	Ranking at the top of the list of 14 caffeinated blends -- earning a
	rating of ``good'' -- are the Starbucks House Blend, calculated at 26
	cents per cup, and Green Mountain Signature Nantucket Blend Medium
	Roast, at 23 cents per cup.

	Blends are the best-selling type of ground coffee and contain beans from
	at least two regions or countries, the publication said.

	The highest score for the 13 decaffeinated coffees also failed to reach
	the top two categories. The better scoring varieties included Allegro
	Organic Decaf, Blend Medium Dark, Peet's Decaf House Blend, Caribou
	Daybreak Coffee Morning Blend Decaf and Bucks County Decaf Breakfast
	blend.

	Consumer Reports has a rating criteria in which the tasters look for
	specific characteristics including the flavor and aroma.

	The publication advised coffee drinkers not to count on familiar brand
	names or expensive price tags, noting that the cost doesn't accurately
	reflect the cost per cup due to varying grind densities, and recommended
	ratios of coffee to water.

	Consumer Reports is published by Consumers Union, an independent
	nonprofit organization that does not accept outside advertising or free
	test samples, it said in a release.

	Full results of the coffee ratings will be available in the March issue
	of Consumer Reports and online at www.ConsumerReports.org.

\end{displayquote}

\subsection{Reason}

While Starbucks was mentioned, this story did not happen in China and was not
related to China or the Chinese.

\section{More good journalists needed despite `fall' of CCTV star}

\begin{itemize}
	\item Author: Ku Ma
	\item Editor: Kong Defang, Yao Chun
	\item Syndicated: Yes, China Daily
\end{itemize}

\begin{displayquote}

	In sharp contrast to the limited coverage of CCTV star anchor Rui
	Chenggang in the official media, social and new media are full of
	reports on his ``being taken away by procuratorate'', with comments
	spreading fast and wide on WeChat.

	Information on Rui's ``being taken away'' is limited, the same as other
	similar cases under investigation. Whether he assists investigation or
	himself is under investigation is not revealed yet. However, it will not
	be a big surprise if Rui turns out to be involved in corruption or
	exploiting his position to make money as some media reports said, given
	the ongoing tremors caused by corruption allegations in the CCTV
	business channel.

	For ordinary Chinese, Rui's ``missing'' was even more dramatic than the
	news of the fall of ``big tiger" Han Xiancong, a top political adviser
	in Anhui province, on the same day. Perhaps the reason for that is that
	35 provincial or higher level ``big tigers'' have already been ensnared
	in the anti-corruption drive since the 18th Congress of the Communist
	Party of China, whereas Rui is the first well-known TV anchor to be
	``taken away''.

	But people need to distinguish his misdeeds, if confirmed, from his
	profession career. Even if Rui is truly under investigation, he is being
	investigated not for his journalistic work, despite its controversial
	nature at times, but for his possible wrongdoings.

	For long, Rui has been admired by youths as a symbol of China's new
	elite generation. He topped the college entrance examination, or gaokao,
	in Hefei, Anhui province, and got admitted to China Foreign Affairs
	University, and ultimately completed his studies at Yale University.
	After joining CCTV, the most influential TV station in China, he has
	interviewed about 200 political and business leaders from across the
	world.

	There is little doubt that Rui possesses almost everything today's
	Chinese youths crave — good education, a well-paying job, handsome
	looks, money and fame. That's perhaps the main reason why his ``being
	taken away'' has dealt a blow to the TV station and sparked heated
	online debates.

	Some critics say Rui deserved to be ``detained'' because he was snobbish
	and arrogant. Others argue that he was ``taken down'' for his
	``nationalist remarks'', and still some others lament his fall from
	grace.

	Also, this is not the first time Rui has created a controversy. In 2007,
	Rui wrote in his microblog that Starbucks should be moved out of China's
	Forbidden City although the coffee shop set up the branch at the
	latter's invitation.  Under the influence of the celebrity who had 10
	million followers in Sina Weibo, China's equivalent Twitter, Starbucks
	eventually moved out of the palace. In that case, people may have quite
	different judgments on whether Rui was too nationalist or despised the
	spirit of contract. However, the modern Starbuck's presence in the
	ancient world heritage itself was controversial, and the Forbidden City
	should have solicited public opinion before the offer was made.

	Later at the G20 meeting in 2010, he raised eyebrows again, this time by
	usurping the last question, supposed to be asked by a South Korean
	journalist, ``to represent the entire Asia'' to ask US President Barack
	Obama .

	The next year, he invited criticism for asking the then US ambassador to
	China, Gary F. Lock, at Davos whether he traveled economy class because
	the US owed huge debts to China.

	His style of raising questions may be too straightforward, or even
	clumsy, for many Chinese who prefer the traditional spirit of politeness
	and modesty. But as a well-educated journalist who speaks English
	fluently, his interview style, despite being controversial, didn't
	affect his work.

	He is one of the few journalists bold enough to challenge powerful
	figures while interviewing them. His success in persuading 200 political
	and business leaders to accept his invitation to be interviewed should
	be considered as an achievement. He had enough reasons to be proud of,
	and he could boast of being the new face of Chinese journalism to the
	outside world.

	His professional achievements and possible illegal deeds are two
	separate things, so his “being taken away” has nothing to do with his
	“patriotic remarks”. Many journalists use cheap tricks to sensationalize
	their reports, because they fear that otherwise viewers and/or readers
	will consider their efforts ordinary and dull. But is it is right to
	sensationalize news even if it is to promote nationalism? More
	journalists need to seriously discuss this issue.

	As a rising power, China is bound to develop deeper relations with the
	international community. So it will need more journalists like Rui to
	reach the country's true story to the outside world. And of course, the
	crackdown on corruption should be intensified to prevent so-called star
	journalists from using their positions to make money.

	\textit{The author is an editor with China Daily.}

\end{displayquote}

\subsection{Reason}

While the article happened in China and Starbucks is mentioned, the story does
not significantly involve the company.
% may need additional review

% vim: textwidth=80 smartindent breakindent
