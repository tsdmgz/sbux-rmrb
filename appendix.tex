\appendix
\chapter{article-grab-link.py}\label{appdx:grab-link}

\inputminted[linenos]{python}{data/script/article-grab-link.py}

\section{Usage}
% Thank you POSIX manpages for content inspiration

\subsection{Synopsis}

article-grab-link.py -t PUBLICATION -f FILE

\subsection{Description}

Parses a downloaded copy of a search results page of a given FILE from
PUBLICATION and returns links to articles in the given FILE.

\subsection{Options}

\begin{description}
	\item [-t PUBLICATION] Use PUBLICATION in parsing a saved results page.
	\item [-f FILE] Obtain links from FILE. Multiple input files are not
	supported.
\end{description}

\subsection{Use case}

In the process of grabbing the articles, a search is performed in People's Daily
website which then returns a results page with the URL
\begin{minted}[breaklines=true]{text}
http://search.people.com.cn/language/english/getResult.jsp
\end{minted}
However, this is not the page we are looking for. The actual URL for the results
page is at the bottom part, at the pagination links and looks like
\begin{minted}[breaklines=true]{text}
http://search.people.com.cn/language/search.do?pageNum=3&keyword=starbucks
&siteName=english&dateFlag=true&a=&b=&c=&d=&e=&f=
\end{minted}

The URL for the results page is now known. As of writing, there are 9 pages in
the results page. A command to download all the pages in the search results can
now be constructed.
% TODO: figure out how to unbreak this somehow
\begin{minted}[breaklines=true]{bash}
cd data/articles/peoples-daily/searchresults
echo "http://search.people.com.cn/language/search.do?pageNum="{1..9}
"&keyword=starbucks&siteName=english&dateFlag=true&a=&b=&c=&d=&e=&f="
 | parallel -n1 -j9 wget -v
\end{minted}
With all pages from the search results downloaded, links to resulting articles
can now be parsed.
\begin{minted}[breaklines=true]{bash}
cd articles
find ../ -maxdepth 1 -type f|parallel ../../../script/article-grab-link.py -t rmrb -f {}|parallel -j30 wget -v {}
\end{minted}
All downloaded articles are downloaded in the current working directory. If the
commands were followed, look in
\mintinline{bash}{data/searchresults/peoples-daily/articles}.

\chapter{Sample included articles}\label{appdx:news-articles-inc}

\section{Former Kuomintang Leader's Old Residence Turned into
McDonald's}

\begin{displayquote}
	Former Kuomintang leader Chiang Ching-kuo's former residence in east China's
	Hangzhou city has been turned into a McDonald's fast food outlet.

	The McDonald's outlet has opened in the main building of the old residence, a
	western-style brick and wood villa built in the 1930s.

	A Starbucks cafe opened two months earlier this year, in a side wing of the
	same building, next to the McDonald's cafe.

	Chiang Ching-kuo was a Kuomintang politician and leader, and son of Chiang
	Kai-shek, who held numerous leadership positions in Taiwan.

	The use of Chiang Ching-kuo's old residence as a coffee shop and fast food
	outlet has stirred concerns about the protection of historical sites among
	netizens.

	An expert on historic preservation says developing historical sites for
	commercial purposes and conducting renovation should both need government
	approval.

	The expert added that those renovating historical sites are not allowed to
	change any interior room structure or exterior building walls.
\end{displayquote}

\subsection{Reason}

Another Starbucks branch has been opened in a historical/cultural site and has
generated controversy, again.

\section{Smoother, faster ride home for Spring Festival}

\begin{displayquote}
	BEIJING - High-speed trains with comfort, Starbucks coffee onboard, free WIFI
	in stations, and phone apps for ticket purchase. As the Spring Festival travel
	rush kicked off on Sunday, hundreds of millions of Chinese found that their
	journeys for holiday homecomings have become much smoother and faster.

	This year's Spring Festival travel rush reflected how China's economic boom,
	huge investment in infrastructure and fast growth of information technologies
	totally redefined the once gruelling experience of going home for the Chinese
	New Year, which falls on February 8 this year.

	MODERNIZED JOURNEY

	At Shanghai Railway Station, the ticket office is no longer crowded. In
	previous years' travel rush, the ticket office was crammed every night with
	tens of thousands of people who had to line up for the whole night to buy a
	ticket.

	But this year, about 83 percent of tickets were purchased online.

	China's railway service has been adapting to hi-tech trends by making itself
	accessible through websites and mobile phone apps, said Zhu Wenzhong, passenger
	traffic director of Shanghai Railway Bureau. Passengers now could order onboard
	meals on the phone app before boarding. Drinks made by Starbucks are available
	on certain trains.

	Across China, free WIFI is offered in some train stations and electronic
	ticketing machines were placed in bus stations. An online system that
	integrates bus operators in 13 provinces has been launched.

	The Ministry of Transport said this year it started to use big data to analyze
	the Spring Festival traffic.

	Chinese car-hailing app Didi rolled out a car-pooling service that can pair
	travelling needs across the country, making it possible for drivers to take on
	others when travelling home for the Chinese Lunar New Year.

	Train stations have also been modernized. In the city of Nanchang, a railway
	hub in east China, passengers used to wait outside Nanchang Railway Station as
	there was not enough room indoors during the Spring Festival travel rush. But
	this year, they can wait inside the station as a high-speed train station was
	just added to the city.

	FASTER RIDE

	This year, Gong Xinyi, a college student in Shanghai, traveled back to her
	hometown in Jiangxi Province with only one third of the time that she used to
	spend.

	A newly added high-speed route has linked Gong's hometown with Shanghai and
	shortened her journey to three hours. Last year, she had to take a
	7-hour-train ride and an additional 3-hour bus trip.

	Gong's faster Spring Festival journey is made possible as China has been
	investing heavily to expand its high-speed train network which is already the
	world's largest.

	Of all the trains serving in the Spring Festival travel rush this year, more
	than 60 percent are high-speed trains that can run up to 350 kilometers per
	hour.

	Around 3,300 kilometers of new lines were added to the high-speed railway
	network last year, bringing the total length to 19,000 kilometers, which make
	up 60 percent of the world's total.

	From 2011 to 2015, the period in which China's 12th Five-Year Plan was
	implemented, fixed-asset investment in railways amounted to 3.58 trillion yuan
	(544 billion U.S. dollars), up 47.3 percent from the 11th five-year-plan
	period.

	Sheng Guangzu, general manager of the China Railway Corp., said China plans to
	invest 800 billion yuan in railways in 2016, especially in less-developed
	central and western regions.

	High-speed rail service continues to carry more weight in the Spring Festival
	travel rush because more Chinese now can afford to travel in style after the
	country's average disposable income surged by more than seven percent every
	year over the last decade.

	Chinese people's growing ability to afford a faster journey has also fueled an
	air travel boom.

	Chinese airlines are expected to carry 54.55 million passengers in the Spring
	Festival travel rush, up 11 percent from the last year.

	Air China said during the Spring Festival travel rush, it would add 2,432
	flights and operate an average of 1,160 flights daily.

	China Southern Airlines planned to add over 6,000 flights on 155 international
	and domestic routes during the Spring Festival travel rush.

	Effort has also been made to ensure a faster trip back on the ground. As no
	toll way fee is charged nationwide during the seven-day Spring Festival
	holiday, traffic jams on highways had trapped thousands of cars for hours in
	previous years.

	Zheng Zongjie, an engineer at the road network center of the Ministry of
	Transport, said this year car drivers would no longer need to pick up the
	tickets at toll gates to make traffic smoother, as they did during previous
	free-hours.

	Urban transportation would also be improved with better arrangement of metro,
	bus and taxi in the Spring Festival travel rush, the authorities said.
\end{displayquote}

\subsection{Reason}

Aside from adopting new tech innovations, it is interesting to note that
Starbucks coffee has been specifically mentioned, when a coach may also have
other food and/or beverage products on board

\chapter{Sample excluded articles}\label{appdx:news-articles-ninc}

% vim: textwidth=80 smartindent breakindent
